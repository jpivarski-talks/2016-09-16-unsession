\documentclass{beamer}

%
% Choose how your presentation looks.
%
% For more themes, color themes and font themes, see:
% http://deic.uab.es/~iblanes/beamer_gallery/index_by_theme.html
%
\mode<presentation>
{
  \usetheme{default}      % or try Darmstadt, Madrid, Warsaw, ...
  \usecolortheme{default} % or try albatross, beaver, crane, ...
  \usefonttheme{default}  % or try serif, structurebold, ...
  \setbeamertemplate{navigation symbols}{}
  \setbeamertemplate{caption}[numbered]
  \setbeamertemplate{footline}[page number]
  \setbeamercolor{frametitle}{fg=white}
  \setbeamercolor{footline}{fg=black}
} 

\usepackage[english]{babel}
\usepackage[utf8x]{inputenc}
\usepackage{tikz}
\usepackage{listings}
\usepackage{courier}

\xdefinecolor{darkblue}{rgb}{0.1,0.1,0.7}
\xdefinecolor{dianablue}{rgb}{0.18,0.24,0.31}
\definecolor{commentgreen}{rgb}{0,0.6,0}
\definecolor{stringmauve}{rgb}{0.58,0,0.82}

\lstset{ %
  backgroundcolor=\color{white},      % choose the background color
  basicstyle=\ttfamily\scriptsize,         % size of fonts used for the code
  breaklines=true,                    % automatic line breaking only at whitespace
  captionpos=b,                       % sets the caption-position to bottom
  commentstyle=\color{commentgreen},  % comment style
  escapeinside={\%*}{*)},             % if you want to add LaTeX within your code
  keywordstyle=\color{blue},          % keyword style
  stringstyle=\color{stringmauve},    % string literal style
  showstringspaces=false,
  showlines=true
}

\lstdefinelanguage{scala}{
  morekeywords={abstract,case,catch,class,def,%
    do,else,extends,false,final,finally,%
    for,if,implicit,import,match,mixin,%
    new,null,object,override,package,%
    private,protected,requires,return,sealed,%
    super,this,throw,trait,true,try,%
    type,val,var,while,with,yield},
  otherkeywords={=>,<-,<\%,<:,>:,\#,@},
  sensitive=true,
  morecomment=[l]{//},
  morecomment=[n]{/*}{*/},
  morestring=[b]",
  morestring=[b]',
  morestring=[b]"""
}

\title[16-09-2016-unsession]{Declarative language on nested structures}
\author{Jim Pivarski}
\institute{Princeton University -- Project DIANA}
\date{September 16, 2016 \\ StrangeLoop Unsession}

\begin{document}

\logo{\pgfputat{\pgfxy(0.11, 8)}{\pgfbox[right,base]{\tikz{\filldraw[fill=dianablue, draw=none] (0 cm, 0 cm) rectangle (50 cm, 1 cm);}}}\pgfputat{\pgfxy(0.11, -0.6)}{\pgfbox[right,base]{\tikz{\filldraw[fill=dianablue, draw=none] (0 cm, 0 cm) rectangle (50 cm, 1 cm);}\includegraphics[height=0.99 cm]{diana-hep-logo.png}\tikz{\filldraw[fill=dianablue, draw=none] (0 cm, 0 cm) rectangle (4.9 cm, 1 cm);}}}}

\begin{frame}
  \titlepage
\end{frame}

\logo{\pgfputat{\pgfxy(0.11, 8)}{\pgfbox[right,base]{\tikz{\filldraw[fill=dianablue, draw=none] (0 cm, 0 cm) rectangle (50 cm, 1 cm);}\includegraphics[height=1 cm]{diana-hep-logo.png}}}}

% Uncomment these lines for an automatically generated outline.
%\begin{frame}{Outline}
%  \tableofcontents
%\end{frame}

\begin{frame}{Motivation}
\vspace{0.5 cm}
In High Energy Physics (HEP), $>$99\% of data analysis code is
\begin{itemize}
\item compiled C++ (modules fitting into a larger framework)
\item {\it interpreted} C++ (don't ask)
\item Python (the young and hip)
\end{itemize}

\vspace{0.5 cm}
But these are fairly regular operations, at least in the large data-reduction phase:
\begin{itemize}
\item filtering events of interest (skimming)
\item selecting variables of interest (slimming)
\item evaluating simple mathematical operations, e.g.
\begin{itemize}
\item invariant mass: $\sqrt{(E_1 + E_2)^2 - |\vec{p}_1 + \vec{p}_2|^2}$
\item min/max of a collection
\item averages, standard deviations, \ldots
\end{itemize}
\end{itemize}

\vspace{0.5 cm}
Could be declarative: simple for users, optimized on the backend!
\end{frame}

\begin{frame}[fragile]{Typical data structure}
\vspace{0.25 cm}
\small
{\bf Tree-like with a known schema.} {\it Maybe} with cross-references (e.g.\ showers matched to tracks), but at least a DAG you can pretend is a tree.
\begin{verbatim}
event: struct
  |
  +--- timestamp: bigint
  +--- missing energy: float
  +--- tracks: array of struct
  |      |
  |      +--- momentum: float
  |      +--- theta angle: float
  |      +--- hits: array of struct
  |             |
  |             +--- detector id: int
  |             +--- charge: float
  |             +--- time: float
  |             +--- ...
  +--- showers: array of struct
         |
         +--- ...
\end{verbatim}
\end{frame}

\begin{frame}{Typical queries}
\vspace{0.25 cm}
{\bf Broad classes:}
\begin{description}
\item[Reduce-like:] transforms a set of records into a single record (maybe different type). \mbox{Aggregations: e.g.\ Histogrammar.\hspace{-1 cm}}
\item[Map-like:] transforms a set of records into another set of records (maybe different type). Looking for a solution.
\end{description}

\vspace{0.25 cm}
{\bf Simplification:} the source dataset is {\it read-only} and is {\it not} updating in real-time. No transactions are taking place.

\vspace{0.25 cm}
{\bf Examples:}
\begin{itemize}
\item ``Momentum of the track with the most hits with theta between $-2.4$ and $2.4$.''
\item ``Average charge of all hits with time in 0--10 ps on all tracks with momentum greater than 10 GeV/c.''
\item ``Weighted average theta of the two tracks with highest momentum.''
\end{itemize}
\end{frame}

\begin{frame}[fragile]{Illustrations in Scala}
\vspace{0.25 cm}
{\tiny
\begin{verbatim}
event: struct
  |
  +--- timestamp: bigint
  +--- missing energy: float
  +--- tracks: array of struct
  |      |
  |      +--- momentum: float
  |      +--- theta angle: float
  |      +--- hits: array of struct
  |             |
  |             +--- detector id: int
  |             +--- charge: float
  |             +--- time: float
  |             +--- ...
  +--- showers: array of struct
         |
         +--- ...
\end{verbatim}}

\begin{itemize}
\item ``Momentum of the track with the most hits with theta between $-2.4$ and $2.4$.''
\end{itemize}

\begin{lstlisting}[language=scala]
// missing check for when zero tracks passed filter!
{event => event.tracks                      // get tracks
               .filter(abs(_.theta) < 2.4)  // theta range
               .maxBy(_.hits.size)          // most hits
               .momentum                    // return momentum
}
\end{lstlisting}
\end{frame}

\begin{frame}[fragile]{Illustrations in Scala}
\vspace{0.35 cm}
\begin{itemize}
\item ``Average charge of all hits with time in 0--10 ps on all tracks with momentum greater than 10 GeV/c.''
\end{itemize}

\begin{lstlisting}[language=scala]
{event => mean(
            event.tracks                    // get tracks
                 .filter(_.momentum > 10)   // in momentum range
                 .flatMap(_.hits)           // explode to hits
                 .filter(_.time < 10)       // in time range
                 .map(_.charge)             // return charges
              )}                            // to mean function
\end{lstlisting}

\begin{itemize}
\item ``Weighted average theta of the two tracks with highest momentum.''
\end{itemize}

\begin{lstlisting}[language=scala]
// again missing check for less than two tracks!
{event => val List(one, two) =              // unpack and assign
              event.tracks                  // get tracks
                   .sortBy(_.momentum)      // sort by momentum
                   .take(2)                 // take two
          // now compute the weighted mean of the two structs
          (one.theta*one.momentum + two.theta*two.momentum) /
              (one.momentum + two.momentum)
}\end{lstlisting}
\end{frame}

\begin{frame}[fragile]{Use SQL?}
\vspace{0.3 cm}
It is valuable to use pre-existing standards if applicable. SQL supports ``arrays of structs,'' can we use it? \uncover<2->{{\bf No.}}

\begin{uncoverenv}<2->
\begin{lstlisting}[language=sql, basicstyle=\ttfamily\scriptsize]
 WITH hit_stats AS (
  SELECT hit.track_id, COUNT(*) AS hit_count FROM hit GROUP BY hit.track_id
 ),
 track_sorted AS (
    SELECT track.*, 
    ROW_NUMBER() OVER (
     PARTITION BY track.event_id
     ORDER BY hit_stats.hit_count DESC
    )
  track_ordinal FROM track 
    INNER JOIN hit_stats ON hit_stats.track_id = track.id
    WHERE track.theta > -2.4 AND track.theta < 2.4
 )

 SELECT * FROM event
   INNER JOIN track_sorted ON track_sorted.event_id = event.id

WHERE
  track_sorted.track_ordinal = 1
\end{lstlisting}
\end{uncoverenv}
\end{frame}

\begin{frame}[fragile]{Use RDF or a graph query language?}
\vspace{0.25 cm}
{\bf Maybe.} This is SPARQL:

\begin{lstlisting}[language=sql]
PREFIX : <http://yournamespace.com/accelerator/> .

SELECT ?momentum (MAX(?hitcount) as ?maxhits)
WHERE {
    SELECT ?momentum (COUNT(?hits) AS ?hitcount)
    WHERE ?track :momentum ?momentum .
          ?track :theta ?theta .
          FILTER (?theta > -2.4 AND ?theta < 2.4) .
          ?track :hits ?hits
    GROUP BY ?track
}
GROUP BY ?momentum;
\end{lstlisting}

\vspace{0.25 cm}
XPath? (Idea from today's Cypher talk\ldots)

\vspace{0.25 cm}
I'm on the fence about this.

\vspace{0.25 cm}
\begin{minipage}{\linewidth}
\scriptsize
RDF is pretty well established and graph query languages are becoming established, but they introduce more concepts than are needed for this problem.

\vspace{0.25 cm}
Having to address unnecessary concepts in every query (e.g.\ saying that you're looking for events with tracks when {\it every} event has tracks) is distracting.
\end{minipage}
\end{frame}

\begin{frame}{What I've been thinking}
\vspace{0.25 cm}
From a user's perspective, I liked the Scala examples best (easily translatable into any other functional syntax).

\vspace{0.25 cm}
But the semantics should also be declarative to develop better backend optimizations.

\vspace{0.25 cm}
\begin{uncoverenv}<2->
\textcolor{darkblue}{\bf Idea:} introduce ``higher order functions'' to SQL?
\begin{itemize}
\item add syntax like {\tt x -> sin(x) + 3*x} to expressions
\item add methods like the following to arrays
\begin{itemize}
\item {\tt \scriptsize COLLECTION.map(function1)}
\item {\tt \scriptsize COLLECTION.reduce(function2)} {\it not} a DAF: operates only on {\it tracks within one event}.
\item {\tt \scriptsize COLLECTION.count}
\item {\tt \scriptsize COLLECTION.sum}
\item {\tt \scriptsize COLLECTION.max(function1, N)} top-N subcollection.
\item {\tt \scriptsize join(COLLECTION1, COLLECTION2)} {\it not} joining events, but {\it tracks within one event} (small).
\item {\tt \scriptsize COLLECTION.pairs} like the above, but unique pairs of indexes.
\end{itemize}
\item add assignments (syntactic sugar).
\end{itemize}
\end{uncoverenv}
\end{frame}

\begin{frame}{}

\end{frame}


\end{document}
